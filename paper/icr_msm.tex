% Options for packages loaded elsewhere
\PassOptionsToPackage{unicode}{hyperref}
\PassOptionsToPackage{hyphens}{url}
%
\documentclass[
]{article}
\usepackage{lmodern}
\usepackage{amssymb,amsmath}
\usepackage{ifxetex,ifluatex}
\ifnum 0\ifxetex 1\fi\ifluatex 1\fi=0 % if pdftex
  \usepackage[T1]{fontenc}
  \usepackage[utf8]{inputenc}
  \usepackage{textcomp} % provide euro and other symbols
\else % if luatex or xetex
  \usepackage{unicode-math}
  \defaultfontfeatures{Scale=MatchLowercase}
  \defaultfontfeatures[\rmfamily]{Ligatures=TeX,Scale=1}
\fi
% Use upquote if available, for straight quotes in verbatim environments
\IfFileExists{upquote.sty}{\usepackage{upquote}}{}
\IfFileExists{microtype.sty}{% use microtype if available
  \usepackage[]{microtype}
  \UseMicrotypeSet[protrusion]{basicmath} % disable protrusion for tt fonts
}{}
\makeatletter
\@ifundefined{KOMAClassName}{% if non-KOMA class
  \IfFileExists{parskip.sty}{%
    \usepackage{parskip}
  }{% else
    \setlength{\parindent}{0pt}
    \setlength{\parskip}{6pt plus 2pt minus 1pt}}
}{% if KOMA class
  \KOMAoptions{parskip=half}}
\makeatother
\usepackage{xcolor}
\IfFileExists{xurl.sty}{\usepackage{xurl}}{} % add URL line breaks if available
\IfFileExists{bookmark.sty}{\usepackage{bookmark}}{\usepackage{hyperref}}
\hypersetup{
  pdftitle={Individual Claim Reserving via Multi-state Models: A Case Study and Recipe},
  pdfauthor={Alexandra Taggart and Rajesh Sahasrabuddhe},
  hidelinks,
  pdfcreator={LaTeX via pandoc}}
\urlstyle{same} % disable monospaced font for URLs
\usepackage[margin=1in]{geometry}
\usepackage{graphicx,grffile}
\makeatletter
\def\maxwidth{\ifdim\Gin@nat@width>\linewidth\linewidth\else\Gin@nat@width\fi}
\def\maxheight{\ifdim\Gin@nat@height>\textheight\textheight\else\Gin@nat@height\fi}
\makeatother
% Scale images if necessary, so that they will not overflow the page
% margins by default, and it is still possible to overwrite the defaults
% using explicit options in \includegraphics[width, height, ...]{}
\setkeys{Gin}{width=\maxwidth,height=\maxheight,keepaspectratio}
% Set default figure placement to htbp
\makeatletter
\def\fps@figure{htbp}
\makeatother
\setlength{\emergencystretch}{3em} % prevent overfull lines
\providecommand{\tightlist}{%
  \setlength{\itemsep}{0pt}\setlength{\parskip}{0pt}}
\setcounter{secnumdepth}{5}

\title{Individual Claim Reserving via Multi-state Models: \n A Case Study and
Recipe}
\author{Alexandra Taggart and Rajesh Sahasrabuddhe}
\date{\today}

\begin{document}
\maketitle

\hypertarget{motivation-and-introduction}{%
\section{Motivation and
Introduction}\label{motivation-and-introduction}}

The most common actuarial models used to develop unpaid claim estimates
require the analysis of claim triangles. The ubiquitous use of triangles
is natural as it provides a natural visualization of the movement of a
\emph{portfolio} of claims. As such, both actuaries and non-actuaries
(i.e., stakeholders) find the claim triangle understandable and, as a
result, acceptable.

However, a claim triangle is ``data-expensive.''

\begin{itemize}
\item
  We require a point-in-time claim listing to construct each diagonal of
  claims triangle and we require several of these point-in-time listing
  evaluated at evenly-spaced intervals. That is, we need to \emph{save}
  significant amounts of data.
\item
  Then we \emph{spend} the data by aggregating the claim-listing by
  cohorts (such as accident year, report year or policy year).
\item
  We \emph{give away} all the features of the claim other than its
  maturity.
\end{itemize}

\hypertarget{modeling-transitions}{%
\section{Modeling Transitions}\label{modeling-transitions}}

This section will focus on state tarnsitions and the minimum outcome
will be probability distrubution as to how claims close.

\hypertarget{defining-states}{%
\subsection{Defining States}\label{defining-states}}

Discuss how the most basic states of ``open'' and ``closed'' and how
model complexity increases as we refine states

\hypertarget{organizing-data}{%
\subsection{Organizing Data}\label{organizing-data}}

Suggest focusing on how to organize data to measure state transitions

\hypertarget{visualizing-a-transition-matrix}{%
\subsection{Visualizing a Transition
Matrix}\label{visualizing-a-transition-matrix}}

Visualizing a transition matrix Parhaps we think about supplementing the
current design with a chart of `off-diagonal' movements.

\hypertarget{selecting-transition-probabilities}{%
\subsection{Selecting Transition
Probabilities}\label{selecting-transition-probabilities}}

Considerations for selecting transition probabilities. Also provide the
code to create a markov-chain object.

\hypertarget{simulation-results}{%
\subsection{Simulation Results}\label{simulation-results}}

Running the simulation and understanding/visualizing the output

\hypertarget{severity}{%
\section{Severity}\label{severity}}

The conditional severity model

\hypertarget{conclusions}{%
\section{Conclusions}\label{conclusions}}

Conclusions and Areas for further reseacrh.

\end{document}
